% !TeX root = report_example.tex
\newcommand*{\MyHeaderPath}{.}% This path definition is also passed to inside the header files.
\newcommand*{\PathToAssets}{../assets}%
\newcommand*{\PathToOutput}{../output/}%
\newcommand*{\PathToOutputTables}{../output/tables}%
% \newcommand*{\PathToBibFile}{bibliography.bib}%


%%%%%%%%%%%%%%%%%%%%%%%%%%%%%%%%%%%%%%
%% This file is compiled with XeLaTex.
%%%%%%%%%%%%%%%%%%%%%%%%%%%%%%%%%%%%%%

\input{\MyHeaderPath/_article_header.tex}
\input{\MyHeaderPath/_lean_header.tex}


\begin{document}
\title{
Do short sellers respond to ESG ratings?
}

\author{
Adriana Mata\footnote{Student - admata@uchicago.edu} \ \& Baptiste Pepin\footnote{Student - baptistepepin@uchicago.edu} \ \& Diego Almau\footnote{Student - diegoalmau@uchicago.edu} \ \& Pranav Phatak\footnote{Student - pphatak@uchicago.edu}
}

\begin{titlepage}
\maketitle

\doublespacing
\begin{abstract}
This study explores the intricate relationship between Environmental, Social, and Governance (ESG) events and securities lending activities, aiming to unveil how these events influence the behavior of short sellers and the broader market. By investigating various ESG incidents, from environmental catastrophes to governance breakdowns, we analyze their impact on securities lending, a proxy for short-selling interest. Our research seeks to determine whether short sellers incorporate ESG considerations into their investment decisions, suggesting a nuanced interplay between ESG events and market dynamics.
\end{abstract}

\end{titlepage}

\doublespacing
\section{Introduction}

This project aims to unravel the complex relationship between Environmental, Social, and Governance (ESG) events and securities lending activities. It aims to dissect the nuanced interplay between ESG considerations and market dynamics, particularly focusing on the strategies and decisions of short sellers in the wake of ESG events. The central thesis of this inquiry probes two critical questions: Is there a discernible relationship between ESG events and securities lending activity, and do short sellers factor ESG events into their investment decisions?

ESG events encompass a broad spectrum of incidents, ranging from environmental catastrophes, such as oil spills and deforestation, to social and governance issues, including labor disputes and corporate governance failures. These events not only have the potential to directly affect a company's financial performance but also impact its stock price indirectly through reputational damage and regulatory actions. Given the financial implications of ESG events, short sellers—investors who bet against the stock prices of companies—may be particularly attuned to these occurrences. Their investment decisions, driven by the anticipation of declining stock prices following ESG mishaps, could significantly influence securities lending activities by altering demand and affecting the terms of securities loans.

By delving into the environmental, social, and governance dimensions of ESG events, this study seeks to shed light on their impact on the securities lending market and the investment strategies of short sellers. The investigation not only contributes to the academic discourse on sustainable finance but also offers practical insights for investors, policymakers, and corporate executives aiming to understand the interconnections between ESG factors and financial market dynamics. Through this lens, the project aspires to illuminate the broader implications of ESG events on market behavior and the strategic considerations of market participants in the context of securities lending.


%\begin{textbox}{green}{Comments}
%Here is a textbox...
%\end{textbox}

%I give an example of a simple table in Table \ref{table:pandas_to_latex_simple_table1.tex}.


%\begin{table}
%\caption{A Simple Table From Pandas, No. 1}
%\centering
%\input{\PathToOutput/pandas_to_latex_simple_table1.tex}
%\caption*{
%  Here I show some data...
%}
%\label{table:pandas_to_latex_simple_table1.tex}
%\end{table}

\newpage
\section{Project Overview}

This project has been challenging from the beginning, as the aim was not merely to replicate a graph or table from an existing study but to create a set of summary statistic metrics to answer a specific question. The absence of a benchmark paper made it difficult to compare our progress and verify if we were on the correct track. Nonetheless, the main objective of our project was to evaluate whether there exists a relationship between an ESG event and securities lending activity, and if short sellers incorporate these ESG events into their investment decisions.

To properly answer the stated question, we needed to pull data from different market data sources, clean it, and generate qualitative and quantitative metrics to conduct a thorough analysis.

Our project incorporates datasets from multiple libraries within WRDS to ensure a comprehensive approach to our investigation. From the RepRisk Library, we sourced data evaluating ESG risks and business conduct, as this incident-based data covers a spectrum of issues, including environmental impacts, labor practices, human rights, supply chain management, corruption, and others. Additionally, this library provides the RepRisk Index (RRI) and RepRisk Rating (RRR), which quantify reputational risk exposure related to ESG issues. Within WRDS, we also drew upon data from the IHS Markit library, which provides independent data, valuations, and trade processing across all asset classes.

Lastly, we extracted data from the Center for Research in Security Prices (CRSP) database, offering historical stock prices, returns, trading volumes, and essential market indicators. The CRSP database is particularly valuable for filling gaps in our dataset, especially concerning shares outstanding. This information is pivotal for accurately computing the financial ratios crucial to our analysis.

The section of the project where we faced the most challenges was during the data merging process. Since the data had been extracted from three different data sources, the tables pulled did not contain the same column names, the companies were referred to differently, and there were no common identifiers to match the rows within each other. This situation led to multiple errors, data mismatches, and the appearance of missing values.

The first merging step was to combine the Markit data with the CRSP data. This merging was facilitated by aligning the dates and the CUSIP 8-digit identifier, a common field in both datasets. This merging process was not overly complicated since there existed an ID column useful for matching both DataFrames. However, the problem arose when we attempted to merge the RepRisk dataset.

The RepRisk dataset, which contained a lot of missing information, significantly complicated the merging process. Similar to the CRSP and Markit data, RepRisk had financial identifiers like CUSIP or ISIN but lacked the direct financial market data linkage, making it challenging to align with the other datasets on a company-by-company basis.

Despite these challenges in data merging, the section of the project where we succeeded was in extracting meaningful results for each of the key ratios from the complex and diverse datasets we managed to compile. By painstakingly cleaning, aligning, and merging the data, we were able to conduct a comprehensive analysis that illuminated the relationship between ESG events and securities lending activities.



\newpage
\section{Analysis}

The following analysis aims to empirically examine whether there is a quantifiable relationship between ESG events and changes in securities lending activity, such as variations in short interest ratios, loan supply ratios, and loan fees. By analyzing data on ESG incidents and securities lending metrics, the study seeks to determine if and how short sellers incorporate ESG considerations into their investment decisions, potentially using ESG events as indicators to guide their short-selling strategies.  In the realm of securities lending, several key metrics are pivotal for assessing market dynamics and investor sentiment. Below, we delve into the critical ratios and explore additional variables of interest within the Markit dataset.

To systematically unpack the influence of ESG events on securities lending, we will present an array of summary statistics for each specified ratio. These statistics encompass the 10th percentile (p10), 25th percentile (p25), median (p50), 75th percentile (p75), 90th percentile (p90), mean, standard deviation (SD), and the overall sample size (N), catering to a granular analysis of each ESG characteristic level. This structured approach allows for a detailed exploration of the nuances in how varying ESG incidents might sway the decisions of short sellers and, by extension, impact the broader securities lending market. Through this lens, our investigation strives to offer a refined perspective on the intersection between ESG considerations and financial market dynamics, contributing valuable insights to both academic discourse and practical investment strategies.


\subsection{Short Interest Ratio}

This ratio, a critical indicator of market sentiment towards a company, provides insights into the proportion of shares borrowed for short selling relative to the total shares available in the market. A higher Short Interest Ratio is often interpreted as a bearish signal, suggesting a heightened interest in short selling possibly due to negative perceptions or expectations about the company's future performance.
 \begin{equation}
	\text{Short Interest Ratio} = \frac{\text{Shares on Loan}}{\text{Shares Outstanding}} \quad \text{or} \quad \frac{\text{QuantityOnLoan}}{\text{SHROUT}}
	\label{eq:Short_Interest_Ratio}
\end{equation}

Each table delineated below serves as a comprehensive summary of statistical analyses, illustrating how different facets of ESG events correlate with changes in the Short Interest Ratio.

\begin{table}[H]
\caption{Summary Stats for Short Interest Ratio for Environmental Level}
\centering
\input{\PathToOutputTables/short_interest_ratio_environment.tex}
\label{table:short_interest_ratio_environment.tex}
\end{table}

\begin{table}[H]
\caption{Summary Stats for Short Interest Ratio for Social Level}
\centering
\input{\PathToOutputTables/short_interest_ratio_social.tex}
\label{table:short_interest_ratio_social.tex}
\end{table}

\begin{table}[H]
\caption{Summary Stats for Short Interest Ratio for Governance Level}
\centering
\input{\PathToOutputTables/short_interest_ratio_governance.tex}
\label{table:short_interest_ratio_governance.tex}
\end{table}


\begin{table}[H]
\caption{Summary Stats for Short Interest Ratio for Novelty Level}
\centering
\input{\PathToOutputTables/short_interest_ratio_novelty.tex}
\label{table:short_interest_ratio_novelty.tex}
\end{table}

\begin{table}[H]
\caption{Summary Stats for Short Interest Ratio for Reach Level}
\centering
\input{\PathToOutputTables/short_interest_ratio_reach.tex}
\label{table:short_interest_ratio_reach.tex}
\end{table}

\begin{table}[H]
\caption{Summary Stats for Short Interest Ratio for Severity Level}
\centering
\input{\PathToOutputTables/short_interest_ratio_severity.tex}
\label{table:short_interest_ratio_severity.tex}
\end{table}


%%%%%%%%%%%%%%%%%%%%%%%%%%%%%%%%%%%%%%%%%%%%%%%%%%%%%%%%%%%%%%%%%%%%%%%%
\subsection{Loan Supply Ratio}


This metric measures the availability of shares for lending against the total shares outstanding. It reflects the willingness of shareholders to lend their shares for short selling, indicating the liquidity and accessibility of shares for short sellers.
 \begin{equation}
	\text{Loan Supply Ratio} = \frac{\text{Shares Available to be Lent}}{\text{Shares Outstanding}} \quad \text{or} \quad \frac{\text{LendableQuantity}}{\text{SHROUT}}
	\label{eq:Loan_Supply_Ratio}
\end{equation}

By presenting these tables, we aim to shed light on the nuanced ways in which the Loan Supply Ratio is affected by different facets of ESG events.

\begin{table}[H]
\caption{Summary Stats for Loan Supply Ratio for Environmental Level}
\centering
\input{\PathToOutputTables/loan_supply_ratio_environment.tex}
\label{table:loan_supply_ratio_environment.tex}
\end{table}

\begin{table}[H]
\caption{Summary Stats for Loan Supply Ratio for Social Level}
\centering
\input{\PathToOutputTables/loan_supply_ratio_social.tex}
\label{table:loan_supply_ratio_social.tex}
\end{table}

\begin{table}[H]
\caption{Summary Stats for Loan Supply Ratio for Governance Level}
\centering
\input{\PathToOutputTables/loan_supply_ratio_governance.tex}
\label{table:loan_supply_ratio_governance.tex}
\end{table}


\begin{table}[H]
\caption{Summary Stats for Loan Supply Ratio for Novelty Level}
\centering
\input{\PathToOutputTables/loan_supply_ratio_novelty.tex}
\label{table:loan_supply_ratio_novelty.tex}
\end{table}

\begin{table}[H]
\caption{Summary Stats for Loan Supply Ratio for Reach Level}
\centering
\input{\PathToOutputTables/loan_supply_ratio_reach.tex}
\label{table:loan_supply_ratio_reach.tex}
\end{table}

\begin{table}[H]
\caption{Summary Stats for Loan Supply Ratio for Severity Level}
\centering
\input{\PathToOutputTables/loan_supply_ratio_severity.tex}
\label{table:loan_supply_ratio_severity.tex}
\end{table}



%%%%%%%%%%%%%%%%%%%%%%%%%%%%%%%%%%%%%%%%%%%%%%%%%%%%%%%%%%%%%%%%%%%%%%%%
\subsection{Loan Utilisation Ratio}

According to the Markit Data Dictionary, this variable is computed as "the value of assets on loan from lenders divided by the total lendable value".


 \begin{equation}
	\text{Loan Utilization Ratio} = \text{Utilisation}
	\label{eq:Loan_Utilization_Ratio}
\end{equation}

By presenting these tables, we aim to shed light on the nuanced ways in which the Loan Supply Ratio is affected by different facets of ESG events.

\begin{table}[H]
\caption{Summary Stats for Loan Utilisation Ratio for Environmental Level}
\centering
\input{\PathToOutputTables/loan_utilisation_ratio_environment.tex}
\label{table:loan_utilisation_ratio_environment.tex}
\end{table}

\begin{table}[H]
\caption{Summary Stats for Loan Utilisation Ratio for Social Level}
\centering
\input{\PathToOutputTables/loan_utilisation_ratio_social.tex}
\label{table:loan_utilisation_ratio_social.tex}
\end{table}

\begin{table}[H]
\caption{Summary Stats for Loan Utilisation Ratio for Governance Level}
\centering
\input{\PathToOutputTables/loan_utilisation_ratio_governance.tex}
\label{table:loan_utilisation_ratio_governance.tex}
\end{table}


\begin{table}[H]
\caption{Summary Stats for Loan Utilisation Ratio for Novelty Level}
\centering
\input{\PathToOutputTables/loan_utilisation_ratio_novelty.tex}
\label{table:loan_utilisation_ratio_novelty.tex}
\end{table}

\begin{table}[H]
\caption{Summary Stats for Loan Utilisation Ratio for Reach Level}
\centering
\input{\PathToOutputTables/loan_utilisation_ratio_reach.tex}
\label{table:loan_utilisation_ratio_reach.tex}
\end{table}

\begin{table}[H]
\caption{Summary Stats for Loan Utilisation Ratio for Severity Level}
\centering
\input{\PathToOutputTables/loan_utilisation_ratio_severity.tex}
\label{table:loan_utilisation_ratio_severity.tex}
\end{table}


%%%%%%%%%%%%%%%%%%%%%%%%%%%%%%%%%%%%%%%%%%%%%%%%%%%%%%%%%%%%%%%%%%%%%%%%
\subsection{Loan Fee Ratio}

The following sections are dedicated to analyzing the Loan Utilization Ratio, a critical measure that assesses the balance between the supply and demand of loanable shares within the securities lending market. This ratio serves as an essential indicator of market sentiment, where a higher utilization rate typically signals robust demand for borrowing shares, often a precursor to increased short selling activity.
 \begin{equation}
	\text{Loan Fee} = \text{IndicativeFee}
	\label{eq:Loan_Fee}
\end{equation}

The introduction of these tables aims to elucidate the relationship between ESG events and the Loan Fee, providing a nuanced perspective on how environmental and social factors not only affect the demand for borrowing shares but also how they can lead to variations in the associated costs.

\begin{table}[H]
\caption{Summary Stats for Loan Fee Ratio for Environmental Level}
\centering
\input{\PathToOutputTables/loan_fee_environment.tex}
\label{table:loan_fee_environment.tex}
\end{table}

\begin{table}[H]
\caption{Summary Stats for Loan Fee Ratio for Social Level}
\centering
\input{\PathToOutputTables/loan_fee_social.tex}
\label{table:loan_fee_social.tex}
\end{table}

\begin{table}[H]
\caption{Summary Stats for Loan Fee Ratio for Governance Level}
\centering
\input{\PathToOutputTables/loan_fee_governance.tex}
\label{table:loan_fee_governance.tex}
\end{table}


\begin{table}[H]
\caption{Summary Stats for Loan Fee Ratio for Novelty Level}
\centering
\input{\PathToOutputTables/loan_fee_novelty.tex}
\label{table:loan_fee_novelty.tex}
\end{table}

\begin{table}[H]
\caption{Summary Stats for Loan Fee Ratio for Reach Level}
\centering
\input{\PathToOutputTables/loan_fee_reach.tex}
\label{table:loan_fee_reach.tex}
\end{table}

\begin{table}[H]
\caption{Summary Stats for Loan Fee Ratio for Severity Level}
\centering
\input{\PathToOutputTables/loan_fee_severity.tex}
\label{table:loan_fee_severity.tex}
\end{table}




%%%%%%%%%%%%%%%%%%%%%%%%%%%%%%%%%%%%%%%%%%%%%%%%%%%%%%%%%%%%%%%%%%%%%%%%
\subsection{Ratios Application}

In this section we are going to analyze the evolution of the lending indicators for Apple Inc. over time. We will use the following ratios: Short Interest Ratio, Loan Supply Ratio, Loan Utilization Ratio, and Loan Fee Ratio. The data is presented in the following figure.

\begin{figure}[H]
\centering
\caption{Lending Indicators for Apple Inc.}
  \centering
  \includegraphics[width=1\linewidth]{\PathToOutput/Apple Inc_lend_ind.png}
\label{fig:apple_lending_indicators}
\end{figure}

Let's break down the analysis of the previous plot into each subplot.

\begin{itemize}
	\item Short Interest Ratio: While displaying moderate fluctuations, has notable inflections that may correspond with ESG-related events. These changes imply a varying degree of shareholder willingness to lend out shares for short selling, potentially reflecting a higher liquidity or investor confidence.
	\item Loan Supply Ratio: It can be noted that it remains relatively stable over the observed period and with a tendency to the upside, suggesting a consistent availability of Apple's shares for lending. This stability is indicative of that significant changes in shares outstanding are uncommon, not generating high spikes in short selling. However, the movements within the ratio may reflect strategic financial decisions by the company, such as share buybacks or issuance of new shares, which can affect the numerator and denominator of the ratio respectively.
	\item Loan Utilization Ratio: It can be noted the higher volatility it displays when compared with the other indicators, with pronounced peaks suggesting moments of increased demand for borrowing shares. These peaks are often aligned with short-selling activities, where investors anticipate a decline in stock prices. Further studies can be done on the relationship between this peaks and valleys and wether they are caused by the usual market events such as earnings reports, product launches, or publications of macroeconomic indicators or ESG-releted events.
	\item Loan Fee Ratio: For Apple Inc, this is the indicator that seems the most stationary at first look. However, out of all of the ratios computed for Apple Inc, the loan fee seems to follow a process similar to a random walk, with no clear trend or seasonality.
\end{itemize}


\begin{figure}[H]
	\centering
	\caption{Lending Indicators for GameStop Corp.}
	  \centering
	  \includegraphics[width=1\linewidth]{\PathToOutput/GameStop Corp_lend_ind.png}
	\label{fig:gamestop_lending_indicators}
\end{figure}

GameStop Corp. is a company that commercializes video games, consumer electronics and gaming merchandise in the retail industry.

In 2021, the stock of the company suffered a so called Short Squeeze, that resulted in an increase of around 1,500\% in the price of the stock. Although this phenomenon was not caused directly by ESG- related events, the computed lending metrics are going to be analyzed to assess how the behave in such an unusual occurrence like this one.

\begin{itemize}
	\item Short Interest Ratio: fluctuates significantly, with a high peak reaching a moment when more than 30\% of the company's Shares Outstanding where being shorted in the marked. These abrupt changes may correlate with the social-media-driven trading surges that GameStop has experienced, contrasting with the more tempered fluctuations seen in Apple's ratio.
	\item Loan Supply Ratio: the graph shows a trend that, although progressively increasing, includes sharp movements both upward and downward. These movements are indicative of a more dynamic change in the availability of shares for lending compared to Apple's relatively stable pattern. This could be a reflection of GameStop's more volatile stock price and the rapid changes in investor behavior that have characterized its trading patterns in recent years.
	\item Loan Utilization Ratio: exhibits high volatility, but most importantly, its level moves in the range between 100 and 50, making an important distinction when compared to the Loan Utilization Ratio from Apple, that stayed below 1.
	\item Loan Fee Ratio: stands out with its pronounced spike that declines over time, diverging from Apple's more random walk-like behavior. This pattern suggests that borrowing GameStop shares occasionally became particularly expensive, which could be due to limited supply or a surge in demand, likely related to the same speculative trading that affects its Short Interest and Loan Utilization Ratios.
\end{itemize}

It is also important not only to compare the trends and volatilities but also the levels of the ratios. While Apple's Short Interest Ratio fluctuated between 0.003 and 0.0005, GameStop's Short Interest Ratio fluctuated between the 0.3 - 0.1 levels. Moreover, the drastic difference in the Loan Utilization Ratio which indicated that the total value of assets on loan from lenders was lower than the total lendable value of the asset, which made sense for such a stable company like Apple Inc.







%%%%%%%%%%%%%%%%%%%%%%%%%%%%%%%%%%%%%%%%%%%%%%%%%%%%%%%%%%%%%%%%%%%%%%%%
\doublespacing
\section{Conclusion}






\end{document}
