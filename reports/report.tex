% !TeX root = report_example.tex
\newcommand*{\MyHeaderPath}{.}% This path definition is also passed to inside the header files.
\newcommand*{\PathToAssets}{../assets}%
\newcommand*{\PathToOutput}{../output/}%
\newcommand*{\PathToOutputTables}{../output/tables}%
% \newcommand*{\PathToBibFile}{bibliography.bib}%


%%%%%%%%%%%%%%%%%%%%%%%%%%%%%%%%%%%%%%
%% This file is compiled with XeLaTex.
%%%%%%%%%%%%%%%%%%%%%%%%%%%%%%%%%%%%%%

\input{\MyHeaderPath/_article_header.tex}
\input{\MyHeaderPath/_lean_header.tex}


\begin{document}
\title{
Do short sellers respond to ESG ratings?
}

\author{
Adriana Mata\footnote{Student - admata@uchicago.edu} \ \& Baptiste Pepin\footnote{Student - baptistepepin@uchicago.edu} \ \& Diego Almau\footnote{Student - diegoalmau@uchicago.edu} \ \& Pranav Phatak\footnote{Student - pphatak@uchicago.edu}
}

\begin{titlepage}
\maketitle

\doublespacing
\begin{abstract}
This study explores the intricate relationship between Environmental, Social, and Governance (ESG) events and securities lending activities, aiming to unveil how these events influence the behavior of short sellers and the broader market. By investigating various ESG incidents, from environmental catastrophes to governance breakdowns, we analyze their impact on securities lending, a proxy for short-selling interest. Our research seeks to determine whether short sellers incorporate ESG considerations into their investment decisions, suggesting a nuanced interplay between ESG events and market dynamics.
\end{abstract}

\end{titlepage}

\doublespacing
\section{Introduction}

This project aims to unravel the complex relationship between Environmental, Social, and Governance (ESG) events and securities lending activities. It aims to dissect the nuanced interplay between ESG considerations and market dynamics, particularly focusing on the strategies and decisions of short sellers in the wake of ESG events. The central thesis of this inquiry probes two critical questions: Is there a discernible relationship between ESG events and securities lending activity, and do short sellers factor ESG events into their investment decisions?

ESG events encompass a broad spectrum of incidents, ranging from environmental catastrophes, such as oil spills and deforestation, to social and governance issues, including labor disputes and corporate governance failures. These events not only have the potential to directly affect a company's financial performance but also impact its stock price indirectly through reputational damage and regulatory actions. Given the financial implications of ESG events, short sellers—investors who bet against the stock prices of companies—may be particularly attuned to these occurrences. Their investment decisions, driven by the anticipation of declining stock prices following ESG mishaps, could significantly influence securities lending activities by altering demand and affecting the terms of securities loans.

By delving into the environmental, social, and governance dimensions of ESG events, this study seeks to shed light on their impact on the securities lending market and the investment strategies of short sellers. The investigation not only contributes to the academic discourse on sustainable finance but also offers practical insights for investors, policymakers, and corporate executives aiming to understand the interconnections between ESG factors and financial market dynamics. Through this lens, the project aspires to illuminate the broader implications of ESG events on market behavior and the strategic considerations of market participants in the context of securities lending.


%\begin{textbox}{green}{Comments}
%Here is a textbox...
%\end{textbox}

%I give an example of a simple table in Table \ref{table:pandas_to_latex_simple_table1.tex}.


%\begin{table}
%\caption{A Simple Table From Pandas, No. 1}
%\centering
%\input{\PathToOutput/pandas_to_latex_simple_table1.tex}
%\caption*{
%  Here I show some data...
%}
%\label{table:pandas_to_latex_simple_table1.tex}
%\end{table}

\newpage
\section{Project Overview}

This project has been challenging from the beginning, as the aim was not merely to replicate a graph or table from an existing study but to create a set of summary statistic metrics to answer a specific question. The absence of a benchmark paper made it difficult to compare our progress and verify if we were on the correct track. Nonetheless, the main objective of our project was to evaluate whether there exists a relationship between an ESG event and securities lending activity, and if short sellers incorporate these ESG events into their investment decisions.

To properly answer the stated question, we needed to pull data from different market data sources, clean it, and generate qualitative and quantitative metrics to conduct a thorough analysis.

Our project incorporates datasets from multiple libraries within WRDS to ensure a comprehensive approach to our investigation. From the RepRisk Library, we sourced data evaluating ESG risks and business conduct, as this incident-based data covers a spectrum of issues, including environmental impacts, labor practices, human rights, supply chain management, corruption, and others. Additionally, this library provides the RepRisk Index (RRI) and RepRisk Rating (RRR), which quantify reputational risk exposure related to ESG issues. Within WRDS, we also drew upon data from the IHS Markit library, which provides independent data, valuations, and trade processing across all asset classes.

Lastly, we extracted data from the Center for Research in Security Prices (CRSP) database, offering historical stock prices, returns, trading volumes, and essential market indicators. The CRSP database is particularly valuable for filling gaps in our dataset, especially concerning shares outstanding. This information is pivotal for accurately computing the financial ratios crucial to our analysis.

The section of the project where we faced the most challenges was during the data merging process. Since the data had been extracted from three different data sources, the tables pulled did not contain the same column names, the companies were referred to differently, and there were no common identifiers to match the rows within each other. This situation led to multiple errors, data mismatches, and the appearance of missing values.

The first merging step was to combine the Markit data with the CRSP data. This merging was facilitated by aligning the dates and the CUSIP 8-digit identifier, a common field in both datasets. This merging process was not overly complicated since there existed an ID column useful for matching both DataFrames. However, the problem arose when we attempted to merge the RepRisk dataset.

The RepRisk dataset, which contained a lot of missing information, significantly complicated the merging process. Similar to the CRSP and Markit data, RepRisk had financial identifiers like CUSIP or ISIN but lacked the direct financial market data linkage, making it challenging to align with the other datasets on a company-by-company basis.

Despite these challenges in data merging, the section of the project where we succeeded was in extracting meaningful results for each of the key ratios from the complex and diverse datasets we managed to compile. By painstakingly cleaning, aligning, and merging the data, we were able to conduct a comprehensive analysis that illuminated the relationship between ESG events and securities lending activities.



\newpage
\section{Analysis}

The following analysis aims to empirically examine whether there is a quantifiable relationship between ESG events and changes in securities lending activity, such as variations in short interest ratios, loan supply ratios, and loan fees. By analyzing data on ESG incidents and securities lending metrics, the study seeks to determine if and how short sellers incorporate ESG considerations into their investment decisions, potentially using ESG events as indicators to guide their short-selling strategies.  In the realm of securities lending, several key metrics are pivotal for assessing market dynamics and investor sentiment. Below, we delve into the critical ratios and explore additional variables of interest within the Markit dataset.

To systematically unpack the influence of ESG events on securities lending, we will present an array of summary statistics for each specified ratio. These statistics encompass the mean, 10th percentile (p10), 25th percentile (p25), median (p50), 75th percentile (p75), 90th percentile (p90), mean, standard deviation (SD), and the overall sample size (N), catering to a granular analysis of each ESG characteristic level. This structured approach allows for a detailed exploration of the nuances in how varying ESG incidents might sway the decisions of short sellers and, by extension, impact the broader securities lending market. Through this lens, our investigation strives to offer a refined perspective on the intersection between ESG considerations and financial market dynamics, contributing valuable insights to both academic discourse and practical investment strategies.

In the following paragrahps, we will explain the conceptual definitions of the ESG events:

\begin{itemize}
	\item\textbf{Environmental Events:} Environmental events can include incidents such as oil spills, deforestation, or emissions scandals. These events often lead to public backlash, regulatory scrutiny, and potentially significant financial impacts on the involved companies. Short sellers, anticipating a drop in stock prices due to these negative outcomes, may increase their activities. Consequently, the demand for borrowing shares of such companies may rise, affecting the securities lending market by increasing loan fees and utilization rates.
	\item\textbf{Social Events:} Social events encompass issues like labor disputes, violations of human rights, and poor working conditions. These incidents can harm a company's reputation, leading to consumer boycotts or loss of investor confidence. Short sellers might view companies embroiled in social controversies as more likely to experience stock price declines, prompting an uptick in short selling activity. This dynamic can alter the supply-demand equilibrium in the securities lending market for the shares of the affected companies.
	\item\textbf{Governance Events:} Governance events involve instances of poor management practices, corruption, executive misconduct, or lack of accountability. Such governance failures can erode investor trust and lead to financial penalties, impacting a company's stock performance. Short sellers often monitor these governance indicators as predictors of potential stock depreciation, influencing their investment strategies and, by extension, the securities lending market through variations in borrowing demand.
\end{itemize}

\subsection{Short Interest Ratio}

This ratio, a critical indicator of market sentiment towards a company, provides insights into the proportion of shares borrowed for short selling relative to the total shares available in the market. A higher Short Interest Ratio is often interpreted as a bearish signal, suggesting a heightened interest in short selling possibly due to negative perceptions or expectations about the company's future performance.

 \begin{equation}
	\text{Short Interest Ratio} = \frac{\text{Shares on Loan}}{\text{Shares Outstanding}} \quad \text{or} \quad \frac{\text{QuantityOnLoan}}{\text{SHROUT}}
	\label{eq:Short_Interest_Ratio}
\end{equation}

Each table delineated below serves as a comprehensive summary of statistical analyses, illustrating how different facets of ESG events correlate with changes in the Short Interest Ratio.

\begin{table}[H]
\caption{Summary Stats for Short Interest Ratio for Environmental Level}
\centering
\resizebox{\textwidth}{!}{
\input{\PathToOutputTables/short_interest_ratio_environment.tex}
}
\label{table:short_interest_ratio_environment.tex}
\end{table}

\begin{table}[H]
\caption{Summary Stats for Short Interest Ratio for Social Level}
\centering
\resizebox{\textwidth}{!}{
\input{\PathToOutputTables/short_interest_ratio_social.tex}
}
\label{table:short_interest_ratio_social.tex}
\end{table}

\begin{table}[H]
\caption{Summary Stats for Short Interest Ratio for Governance Level}
\centering
\resizebox{\textwidth}{!}{
\input{\PathToOutputTables/short_interest_ratio_governance.tex}
}
\label{table:short_interest_ratio_governance.tex}
\end{table}

\begin{table}[H]
\caption{Summary Stats for Short Interest Ratio for Novelty Level}
\centering
\resizebox{\textwidth}{!}{
\input{\PathToOutputTables/short_interest_ratio_novelty.tex}
}
\label{table:short_interest_ratio_novelty.tex}
\end{table}

\begin{table}[H]
\caption{Summary Stats for Short Interest Ratio for Reach Level}
\centering
\resizebox{\textwidth}{!}{
\input{\PathToOutputTables/short_interest_ratio_reach.tex}
}
\label{table:short_interest_ratio_reach.tex}
\end{table}

\begin{table}[H]
\caption{Summary Stats for Short Interest Ratio for Severity Level}
\centering
\resizebox{\textwidth}{!}{
\input{\PathToOutputTables/short_interest_ratio_severity.tex}
}
\label{table:short_interest_ratio_severity.tex}
\end{table}

In this table we are looking at the summary statistics for the Short Interest Ratio while controlling for severity of the ESG risk incident for the entire universe of stocks. That is, a function of the consequences of the severity with respect to ESG issues, the extent of its impact, and its type.

As we would expect, the lower the severity level, the higher the amount of observations for the Short Interest Ratio.

However, when it comes to the mean of the ESG incidents, we re getting counter intuitive results. The results show that the higher the incident severity, the lower the mean Short Interest ratio. This is an interesting result that can have further investigations.

A similar result is found for the standard deviation, the higher the severity, the lower the standard deviation. This is also an interesting result that can have further investigations.

It is interesting to highlight that the highest Short Interest Ratio achieved by any company in the dataframe was achieved when a company had an incident of severity level 1 and it was of 46.57\%. That means, almost half of the available stocks were being borrowed for short selling in that moment.


%%%%%%%%%%%%%%%%%%%%%%%%%%%%%%%%%%%%%%%%%%%%%%%%%%%%%%%%%%%%%%%%%%%%%%%%
\subsection{Loan Supply Ratio}


This metric measures the availability of shares for lending against the total shares outstanding. It reflects the willingness of shareholders to lend their shares for short selling, indicating the liquidity and accessibility of shares for short sellers.
 \begin{equation}
	\text{Loan Supply Ratio} = \frac{\text{Shares Available to be Lent}}{\text{Shares Outstanding}} \quad \text{or} \quad \frac{\text{LendableQuantity}}{\text{SHROUT}}
	\label{eq:Loan_Supply_Ratio}
\end{equation}

By presenting these tables, we aim to shed light on the nuanced ways in which the Loan Supply Ratio is affected by different facets of ESG events.

\begin{table}[H]
\caption{Summary Stats for Loan Supply Ratio for Environmental Level}
\centering
\resizebox{\textwidth}{!}{
\input{\PathToOutputTables/loan_supply_ratio_environment.tex}
}
\label{table:loan_supply_ratio_environment.tex}
\end{table}

\begin{table}[H]
\caption{Summary Stats for Loan Supply Ratio for Social Level}
\centering
\resizebox{\textwidth}{!}{
\input{\PathToOutputTables/loan_supply_ratio_social.tex}
}
\label{table:loan_supply_ratio_social.tex}
\end{table}

\begin{table}[H]
\caption{Summary Stats for Loan Supply Ratio for Governance Level}
\centering
\resizebox{\textwidth}{!}{
\input{\PathToOutputTables/loan_supply_ratio_governance.tex}
}
\label{table:loan_supply_ratio_governance.tex}
\end{table}


\begin{table}[H]
\caption{Summary Stats for Loan Supply Ratio for Novelty Level}
\centering
\resizebox{\textwidth}{!}{
\input{\PathToOutputTables/loan_supply_ratio_novelty.tex}
}
\label{table:loan_supply_ratio_novelty.tex}
\end{table}

\begin{table}[H]
\caption{Summary Stats for Loan Supply Ratio for Reach Level}
\centering
\resizebox{\textwidth}{!}{
\input{\PathToOutputTables/loan_supply_ratio_reach.tex}
}
\label{table:loan_supply_ratio_reach.tex}
\end{table}

\begin{table}[H]
\caption{Summary Stats for Loan Supply Ratio for Severity Level}
\centering
\resizebox{\textwidth}{!}{
\input{\PathToOutputTables/loan_supply_ratio_severity.tex}
}
\label{table:loan_supply_ratio_severity.tex}
\end{table}



%%%%%%%%%%%%%%%%%%%%%%%%%%%%%%%%%%%%%%%%%%%%%%%%%%%%%%%%%%%%%%%%%%%%%%%%
\subsection{Loan Utilisation Ratio}

According to the Markit Data Dictionary, this variable is computed as "the value of assets on loan from lenders divided by the total lendable value".


 \begin{equation}
	\text{Loan Utilisation Ratio} = \text{Utilisation}
	\label{eq:Loan_Utilisation_Ratio}
\end{equation}

By presenting these tables, we aim to shed light on the nuanced ways in which the Loan Supply Ratio is affected by different facets of ESG events.

\begin{table}[H]
\caption{Summary Stats for Loan Utilisation Ratio for Environmental Level}
\centering
\resizebox{\textwidth}{!}{
\input{\PathToOutputTables/loan_utilisation_ratio_environment.tex}
}
\label{table:loan_utilisation_ratio_environment.tex}
\end{table}

As described in the RepRisk data documentation, the core research scope of RepRisk comprises 28 ESG Issues, which are divided into environmental, social, governance, and cross- sectional issues. Specifically, the Environment data element reflects T if the risk incident is linked to an environmental issue, else F.

Therefore, in this case we can appreciate how the maximum Loan Utilisation Ratio achieved by any company when an Enivornmental incident happened was 99.11\%, while when no incident happened, the maximum was 100\%.

Moreover, we see an increas in the mean Loan Utilisation ratio when there occurs an environmental incident, which is a result that favors the thesis that there is an impact in lending metrics when an ESG incident happens.

The meedian of the Loan Utilisation ratio gives us another argument in favor of the thesis explored in this project. We can see an increase from 0.97\% when there is No Environmental incident to 1.61\% when there is an Environmental incident.

Therefore, these results provoke further investigation and analysis to understand the relationship between Environmental incidents and the Loan Utilisation Ratio.

\begin{table}[H]
\caption{Summary Stats for Loan Utilisation Ratio for Social Level}
\centering
\resizebox{\textwidth}{!}{
\input{\PathToOutputTables/loan_utilisation_ratio_social.tex}
}
\label{table:loan_utilisation_ratio_social.tex}
\end{table}

\begin{table}[H]
\caption{Summary Stats for Loan Utilisation Ratio for Governance Level}
\centering
\resizebox{\textwidth}{!}{
\input{\PathToOutputTables/loan_utilisation_ratio_governance.tex}
}
\label{table:loan_utilisation_ratio_governance.tex}
\end{table}


\begin{table}[H]
\caption{Summary Stats for Loan Utilisation Ratio for Novelty Level}
\centering
\resizebox{\textwidth}{!}{
\input{\PathToOutputTables/loan_utilisation_ratio_novelty.tex}
}
\label{table:loan_utilisation_ratio_novelty.tex}
\end{table}

\begin{table}[H]
\caption{Summary Stats for Loan Utilisation Ratio for Reach Level}
\centering
\resizebox{\textwidth}{!}{
\input{\PathToOutputTables/loan_utilisation_ratio_reach.tex}
}
\label{table:loan_utilisation_ratio_reach.tex}
\end{table}

\begin{table}[H]
\caption{Summary Stats for Loan Utilisation Ratio for Severity Level}
\centering
\resizebox{\textwidth}{!}{
\input{\PathToOutputTables/loan_utilisation_ratio_severity.tex}
}
\label{table:loan_utilisation_ratio_severity.tex}
\end{table}


%%%%%%%%%%%%%%%%%%%%%%%%%%%%%%%%%%%%%%%%%%%%%%%%%%%%%%%%%%%%%%%%%%%%%%%%
\subsection{Loan Fee}

The following sections are dedicated to analyzing the Loan Utilisation Ratio, a critical measure that assesses the balance between the supply and demand of loanable shares within the securities lending market. This ratio serves as an essential indicator of market sentiment, where a higher utilisation rate typically signals robust demand for borrowing shares, often a precursor to increased short selling activity.

 \begin{equation}
	\text{Loan Fee} = \text{IndicativeFee}
	\label{eq:Loan_Fee}
\end{equation}

The introduction of these tables aims to elucidate the relationship between ESG events and the Loan Fee, providing a nuanced perspective on how environmental and social factors not only affect the demand for borrowing shares but also how they can lead to variations in the associated costs.

\begin{table}[H]
\caption{Summary Stats for Loan Fee for Environmental Level}
\centering
\resizebox{\textwidth}{!}{
\input{\PathToOutputTables/loan_fee_environment.tex}
}
\label{table:loan_fee_environment.tex}
\end{table}

\begin{table}[H]
\caption{Summary Stats for Loan Fee for Social Level}
\centering
\resizebox{\textwidth}{!}{
\input{\PathToOutputTables/loan_fee_social.tex}
}
\label{table:loan_fee_social.tex}
\end{table}

\begin{table}[H]
\caption{Summary Stats for Loan Fee for Governance Level}
\centering
\resizebox{\textwidth}{!}{
\input{\PathToOutputTables/loan_fee_governance.tex}
}
\label{table:loan_fee_governance.tex}
\end{table}


\begin{table}[H]
\caption{Summary Stats for Loan Fee for Novelty Level}
\centering
\resizebox{\textwidth}{!}{
\input{\PathToOutputTables/loan_fee_novelty.tex}
}
\label{table:loan_fee_novelty.tex}
\end{table}

\begin{table}[H]
\caption{Summary Stats for Loan Fee for Reach Level}
\centering
\resizebox{\textwidth}{!}{
\input{\PathToOutputTables/loan_fee_reach.tex}
}
\label{table:loan_fee_reach.tex}
\end{table}

\begin{table}[H]
\caption{Summary Stats for Loan Fee for Severity Level}
\centering
\resizebox{\textwidth}{!}{
\input{\PathToOutputTables/loan_fee_severity.tex}
}
\label{table:loan_fee_severity.tex}
\end{table}




%%%%%%%%%%%%%%%%%%%%%%%%%%%%%%%%%%%%%%%%%%%%%%%%%%%%%%%%%%%%%%%%%%%%%%%%
\subsection{Ratios Application}

In this section we are going to analyze the evolution of the lending indicators for Apple Inc, GameStop Corp. and Altria Group Inc. over time. We will use the following ratios: Short Interest Ratio, Loan Supply Ratio, Loan Utilisation Ratio, and Loan Fee. The data is presented in the following figure.

\subsubsection{Apple Inc.}

Apple Inc. is a company that designs, produces and commercializes electronic products. We are going to analyze the evolution of Apple's Lending Metrics since as of March 2024, it was the second largest valued company in the US Stock market, just below Microsoft Inc that surpassed it in January of the same year.

Therefore, we can categorize this company as a 'mega cap' company, and an in-between on growth and core in terms of it's categorization in terms of Price to Earnings ratio and other indicators used to build that categorization. For this type of stocks, it is important to note that Short Sellers usually don't have enough market participation in terms of the total shares outstanding to impose important pressures on determining the stock's price.

\begin{figure}[H]
\centering
\caption{Lending Indicators for Apple Inc.}
  \centering
  \includegraphics[width=1\linewidth]{\PathToOutput/Apple Inc_lend_ind.png}
\label{fig:apple_lending_indicators}
\end{figure}

Let's break down the analysis of the previous plot into each subplot.

\begin{itemize}
	\item\textbf{Short Interest Ratio:} While displaying moderate fluctuations, has notable inflections that may correspond with ESG-related events. These changes imply a varying degree of shareholder willingness to lend out shares for short selling, potentially reflecting a higher liquidity or investor confidence.
	\item\textbf{Loan Supply Ratio:} It can be noted that it remains relatively stable over the observed period and with a tendency to the upside, suggesting a consistent availability of Apple's shares for lending. This stability is indicative of that significant changes in shares outstanding are uncommon, not generating high spikes in short selling. However, the movements within the ratio may reflect strategic financial decisions by the company, such as share buybacks or issuance of new shares, which can affect the numerator and denominator of the ratio respectively.
	\item\textbf{Loan Utilisation Ratio:} It can be noted the higher volatility it displays when compared with the other indicators, with pronounced peaks suggesting moments of increased demand for borrowing shares. These peaks are often aligned with short-selling activities, where investors anticipate a decline in stock prices. Further studies can be done on the relationship between this peaks and valleys and wether they are caused by the usual market events such as earnings reports, product launches, or publications of macroeconomic indicators or ESG-releted events.
	\item\textbf{Loan Fee:} For Apple Inc, this is the indicator that seems the most stationary at first look. However, out of all of the ratios computed for Apple Inc, the loan fee seems to follow a process similar to a random walk, with no clear trend or seasonality.
\end{itemize}


\subsubsection{GameStop Corp.}

GameStop Corp. is a company that commercializes video games, consumer electronics and gaming merchandise in the retail industry. We can categorize this stock as a small-cap, core company. Whose business model was starting to get obsolete as the digital video game industry started to rise. Thererfore, as we we will discuss in the following paragraph, it is important to note that there might be market agents that may have an elevated market participation quote and therefore might be able to impact the company's stock price to a certain degree.

In 2021, the stock of the company suffered a so called Short Squeeze, that resulted in an increase of around 1,500\% in the price of the stock. Although this phenomenon was not caused directly by ESG- related events, the computed lending metrics are going to be analyzed to assess how the behave in such an unusual occurrence like this one.

\begin{figure}[H]
	\centering
	\caption{Lending Indicators for GameStop Corp.}
	  \centering
	  \includegraphics[width=1\linewidth]{\PathToOutput/GameStop Corp_lend_ind.png}
	\label{fig:gamestop_lending_indicators}
\end{figure}

Let's break down the analysis of the previous plot into each subplot.

\begin{itemize}
	\item\textbf{Short Interest Ratio:} fluctuates significantly, with a high peak reaching a moment when more than 30\% of the company's Shares Outstanding where being shorted in the marked. These abrupt changes may correlate with the social-media-driven trading surges that GameStop has experienced, contrasting with the more tempered fluctuations seen in Apple's ratio.
	\item\textbf{Loan Supply Ratio:} the graph shows a trend that, although progressively increasing, includes sharp movements both upward and downward. These movements are indicative of a more dynamic change in the availability of shares for lending compared to Apple's relatively stable pattern. This could be a reflection of GameStop's more volatile stock price and the rapid changes in investor behavior that have characterized its trading patterns in recent years.
	\item\textbf{Loan Utilisation Ratio:} exhibits high volatility, but most importantly, its level moves in the range between 100\% and 50\%, making an important distinction when compared to the Loan Utilisation Ratio from Apple, that didn't surpass the 0.7\% level.
	\item\textbf{Loan Fee:} stands out with its pronounced spike that declines over time, diverging from Apple's more random walk-like behavior. This pattern suggests that borrowing GameStop shares occasionally became particularly expensive, which could be due to limited supply or a surge in demand, likely related to the same speculative trading that affects its Short Interest and Loan Utilisation Ratios.
\end{itemize}

It is also important not only to compare the trends and volatilities but also the levels of the ratios. While Apple's Short Interest Ratio fluctuated between 0.3 and 0.05, GameStop's Short Interest Ratio fluctuated between the 0.3 - 0.1 levels. Moreover, the drastic difference in the Loan Utilisation Ratio which indicated that the total value of Apple stocks on loan from lenders was lower than the total lendable value of the Apple stock.

For GameStop, we can appreciate a Loan Utilisation Ratio that almost reaches the 100\% level, reflecting the critical situation of the stock in terms of short seller's interest to open positions on it and it also shows the impact of the Short Squeeze on 2021.


\subsubsection{Altria Group Inc.}

Altria Group Inc. is one of the world's largest producers and commercializers of tobacco, cigarettes and related products. Since this company is involved with the manufacture of tobacco-related products, it is often avoided by investors that are keen on ESG factors, getting explicitly excluded from important tradable ESG investment products such as mutual funds or ETFs.

More specifically, Altria Group Inc is being chosen in this analysis since, starting from 2022, the company was involved in legal issues including the payment of USD 235 million to settle at least 6,000 lawsuits accusing it of fueling a teen vaping epidemic through its former investment in e-cigarette maker Juul Labs Inc.

\begin{figure}[H]
	\centering
	\caption{Lending Indicators for Altria Group Inc.}
	  \centering
	  \includegraphics[width=1\linewidth]{\PathToOutput/Altria Group Inc_lend_ind.png}
	\label{fig:altria_lending_indicators}
\end{figure}

Let's break down the analysis of the previous plot into each subplot.

\begin{itemize}
	\item\textbf{Short Interest Ratio:} The Short Interest Ratio demonstrates significant fluctuations, with sporadic sharp peaks that could be associated with periods of negative news related to the company’s legal challenges or regulatory pressures on vaping products. The variability in this ratio may suggest investor caution, driven by the company's exclusion from ESG investment products and the resulting impact on share liquidity and availability for short sellers.
	\item\textbf{Loan Supply Ratio:} Altria's Loan Supply Ratio shows a descending trend towards the end of the observed period, which may indicate a contraction in the availability of shares for lending. This downward movement might be a reflection of the company's strategic financial activities, such as share repurchases, or a reduced willingness of shareholders to lend their shares amidst the company's legal settlements and the controversy surrounding its investment in vaping products.
	\item\textbf{Loan Utilisation Ratio:} The Loan Utilisation Ratio reveals periods of high volatility with pronounced spikes. These peaks could represent moments of heightened borrowing demand, potentially linked to traders speculating on price decreases in response to legal rulings or adverse public health campaigns against smoking and vaping. The sharp movements suggest that the demand for shorting Altria's stock is influenced by significant external events that impact the company's reputation and regulatory environment.
	\item\textbf{Loan Fee:} exhibits variability with one notable spike to the upside, suggesting one specific period where the cost of borrowing Altria's shares increases substantially, further studies can explore the causation of this spike. These could coincide with increased market uncertainty or speculation related to Altria’s legal and regulatory challenges. The spikes in the Loan Fee might reflect short-term constraints in supply or heightened risk perceived by lenders due to the company's potential financial liabilities and market risks.
\end{itemize}

%%%%%%%%%%%%%%%%%%%%%%%%%%%%%%%%%%%%%%%%%%%%%%%%%%%%%%%%%%%%%%%%%%%%%%%%
\subsection{Changes in Lending Indicators}

In this section we are going to compute the same summary statistics as before but for the changes in the different lending indicators. We are going to look at two different windows: a one-week ahead change and a one-month ahead change.

\subsubsection{Changes in Short Interest Ratio}

\begin{table}[H]
\caption{Summary Stats for changes in Short Interest Ratio for Environmental Level - 1 week ahead}
\centering
\resizebox{\textwidth}{!}{
\input{\PathToOutputTables/short_interest_ratio_environment_change_5.tex}
}
\label{table:short_interest_ratio_environment_change_5.tex}
\end{table}

\begin{table}[H]
\caption{Summary Stats for changes in Short Interest Ratio for Environmental Level - 1 month ahead}
\centering
\resizebox{\textwidth}{!}{
\input{\PathToOutputTables/short_interest_ratio_environment_change_26.tex}
}
\label{table:short_interest_ratio_environment_change_26.tex}
\end{table}

\begin{table}[H]
\caption{Summary Stats for changes in Short Interest Ratio for Social Level - 1 week ahead}
\centering
\resizebox{\textwidth}{!}{
\input{\PathToOutputTables/short_interest_ratio_social_change_5.tex}
}
\label{table:short_interest_ratio_social_change_5.tex}
\end{table}

\begin{table}[H]
\caption{Summary Stats for changes in Short Interest Ratio for Social Level - 1 month ahead}
\centering
\resizebox{\textwidth}{!}{
\input{\PathToOutputTables/short_interest_ratio_social_change_26.tex}
}
\label{table:short_interest_ratio_social_change_26.tex}
\end{table}

\begin{table}[H]
\caption{Summary Stats for changes in Short Interest Ratio for Governance Level - 1 week ahead}
\centering
\resizebox{\textwidth}{!}{
\input{\PathToOutputTables/short_interest_ratio_governance_change_5.tex}
}
\label{table:short_interest_ratio_governance_change_5.tex}
\end{table}

\begin{table}[H]
\caption{Summary Stats for changes in Short Interest Ratio for Governance Level - 1 month ahead}
\centering
\resizebox{\textwidth}{!}{
\input{\PathToOutputTables/short_interest_ratio_governance_change_26.tex}
}
\label{table:short_interest_ratio_governance_change_26.tex}
\end{table}

\begin{table}[H]
\caption{Summary Stats for changes in Short Interest Ratio for Novelty Level - 1 week ahead}
\centering
\resizebox{\textwidth}{!}{
\input{\PathToOutputTables/short_interest_ratio_novelty_change_5.tex}
}
\label{table:short_interest_ratio_novelty_change_5.tex}
\end{table}

\begin{table}[H]
\caption{Summary Stats for changes in Short Interest Ratio for Novelty Level - 1 month ahead}
\centering
\resizebox{\textwidth}{!}{
\input{\PathToOutputTables/short_interest_ratio_novelty_change_26.tex}
}
\label{table:short_interest_ratio_novelty_change_26.tex}
\end{table}

\begin{table}[H]
\caption{Summary Stats for changes in Short Interest Ratio for Reach Level - 1 week ahead}
\centering
\resizebox{\textwidth}{!}{
\input{\PathToOutputTables/short_interest_ratio_reach_change_5.tex}
}
\label{table:short_interest_ratio_reach_change_5.tex}
\end{table}

\begin{table}[H]
\caption{Summary Stats for changes in Short Interest Ratio for Reach Level - 1 month ahead}
\centering
\resizebox{\textwidth}{!}{
\input{\PathToOutputTables/short_interest_ratio_reach_change_26.tex}
}
\label{table:short_interest_ratio_reach_change_26.tex}
\end{table}

\begin{table}[H]
\caption{Summary Stats for changes in Short Interest Ratio for Severity Level - 1 week ahead}
\centering
\resizebox{\textwidth}{!}{
\input{\PathToOutputTables/short_interest_ratio_severity_change_5.tex}
}
\label{table:short_interest_ratio_severity_change_5.tex}
\end{table}

\begin{table}[H]
\caption{Summary Stats for changes in Short Interest Ratio for Severity Level - 1 month ahead}
\centering
\resizebox{\textwidth}{!}{
\input{\PathToOutputTables/short_interest_ratio_severity_change_26.tex}
}
\label{table:short_interest_ratio_severity_change_26.tex}
\end{table}

%%%%%%%%%%%%%%%%%%%%%%%%%%%%%%%%%%%%%%%%%%%%%%%%%%%%%%%%%%%%%%%%%%%%%%%%
\subsubsection{Changes in Loan Supply Ratio}

\begin{table}[H]
\caption{Summary Stats for changes in Loan Supply Ratio for Environmental Level - 1 week ahead}
\centering
\resizebox{\textwidth}{!}{
\input{\PathToOutputTables/loan_supply_ratio_environment_change_5.tex}
}
\label{table:loan_supply_ratio_environment_change_5.tex}
\end{table}

\begin{table}[H]
\caption{Summary Stats for changes in Loan Supply Ratio for Environmental Level - 1 month ahead}
\centering
\resizebox{\textwidth}{!}{
\input{\PathToOutputTables/loan_supply_ratio_environment_change_26.tex}
}
\label{table:loan_supply_ratio_environment_change_26.tex}
\end{table}

\begin{table}[H]
\caption{Summary Stats for changes in Loan Supply Ratio for Social Level - 1 week ahead}
\centering
\resizebox{\textwidth}{!}{
\input{\PathToOutputTables/loan_supply_ratio_social_change_5.tex}
}
\label{table:loan_supply_ratio_social_change_5.tex}
\end{table}

\begin{table}[H]
\caption{Summary Stats for changes in Loan Supply Ratio for Social Level - 1 month ahead}
\centering
\resizebox{\textwidth}{!}{
\input{\PathToOutputTables/loan_supply_ratio_social_change_26.tex}
}
\label{table:loan_supply_ratio_social_change_26.tex}
\end{table}

\begin{table}[H]
\caption{Summary Stats for changes in Loan Supply Ratio for Governance Level - 1 week ahead}
\centering
\resizebox{\textwidth}{!}{
\input{\PathToOutputTables/loan_supply_ratio_governance_change_5.tex}
}
\label{table:loan_supply_ratio_governance_change_5.tex}
\end{table}

\begin{table}[H]
\caption{Summary Stats for changes in Loan Supply Ratio for Governance Level - 1 month ahead}
\centering
\resizebox{\textwidth}{!}{
\input{\PathToOutputTables/loan_supply_ratio_governance_change_26.tex}
}
\label{table:loan_supply_ratio_governance_change_26.tex}
\end{table}

\begin{table}[H]
\caption{Summary Stats for changes in Loan Supply Ratio for Novelty Level - 1 week ahead}
\centering
\resizebox{\textwidth}{!}{
\input{\PathToOutputTables/loan_supply_ratio_novelty_change_5.tex}
}
\label{table:loan_supply_ratio_novelty_change_5.tex}
\end{table}

\begin{table}[H]
\caption{Summary Stats for changes in Loan Supply Ratio for Novelty Level - 1 month ahead}
\centering
\resizebox{\textwidth}{!}{
\input{\PathToOutputTables/loan_supply_ratio_novelty_change_26.tex}
}
\label{table:loan_supply_ratio_novelty_change_26.tex}
\end{table}

\begin{table}[H]
\caption{Summary Stats for changes in Loan Supply Ratio for Reach Level - 1 week ahead}
\centering
\resizebox{\textwidth}{!}{
\input{\PathToOutputTables/loan_supply_ratio_reach_change_5.tex}
}
\label{table:loan_supply_ratio_reach_change_5.tex}
\end{table}

\begin{table}[H]
\caption{Summary Stats for changes in Loan Supply Ratio for Reach Level - 1 month ahead}
\centering
\resizebox{\textwidth}{!}{
\input{\PathToOutputTables/loan_supply_ratio_reach_change_26.tex}
}
\label{table:loan_supply_ratio_reach_change_26.tex}
\end{table}

\begin{table}[H]
\caption{Summary Stats for changes in Loan Supply Ratio for Severity Level - 1 week ahead}
\centering
\resizebox{\textwidth}{!}{
\input{\PathToOutputTables/loan_supply_ratio_severity_change_5.tex}
}
\label{table:loan_supply_ratio_severity_change_5.tex}
\end{table}

\begin{table}[H]
\caption{Summary Stats for changes in Loan Supply Ratio for Severity Level - 1 month ahead}
\centering
\resizebox{\textwidth}{!}{
\input{\PathToOutputTables/loan_supply_ratio_severity_change_26.tex}
}
\label{table:loan_supply_ratio_severity_change_26.tex}
\end{table}

%%%%%%%%%%%%%%%%%%%%%%%%%%%%%%%%%%%%%%%%%%%%%%%%%%%%%%%%%%%%%%%%%%%%%%%%
\subsubsection{Changes in Loan Utilisation Ratio}

\begin{table}[H]
\caption{Summary Stats for changes in Loan Utilisation Ratio for Environmental Level - 1 week ahead}
\centering
\resizebox{\textwidth}{!}{
\input{\PathToOutputTables/loan_utilisation_ratio_environment_change_5.tex}
}
\label{table:loan_utilisation_ratio_environment_change_5.tex}
\end{table}

\begin{table}[H]
\caption{Summary Stats for changes in Loan Utilisation Ratio for Environmental Level - 1 month ahead}
\centering
\resizebox{\textwidth}{!}{
\input{\PathToOutputTables/loan_utilisation_ratio_environment_change_26.tex}
}
\label{table:loan_utilisation_ratio_environment_change_26.tex}
\end{table}

\begin{table}[H]
\caption{Summary Stats for changes in Loan Utilisation Ratio for Social Level - 1 week ahead}
\centering
\resizebox{\textwidth}{!}{
\input{\PathToOutputTables/loan_utilisation_ratio_social_change_5.tex}
}
\label{table:loan_utilisation_ratio_social_change_5.tex}
\end{table}

\begin{table}[H]
\caption{Summary Stats for changes in Loan Utilisation Ratio for Social Level - 1 month ahead}
\centering
\resizebox{\textwidth}{!}{
\input{\PathToOutputTables/loan_utilisation_ratio_social_change_26.tex}
}
\label{table:loan_utilisation_ratio_social_change_26.tex}
\end{table}

\begin{table}[H]
\caption{Summary Stats for changes in Loan Utilisation Ratio for Governance Level - 1 week ahead}
\centering
\resizebox{\textwidth}{!}{
\input{\PathToOutputTables/loan_utilisation_ratio_governance_change_5.tex}
}
\label{table:loan_utilisation_ratio_governance_change_5.tex}
\end{table}

\begin{table}[H]
\caption{Summary Stats for changes in Loan Utilisation Ratio for Governance Level - 1 month ahead}
\centering
\resizebox{\textwidth}{!}{
\input{\PathToOutputTables/loan_utilisation_ratio_governance_change_26.tex}
}
\label{table:loan_utilisation_ratio_governance_change_26.tex}
\end{table}

\begin{table}[H]
\caption{Summary Stats for changes in Loan Utilisation Ratio for Novelty Level - 1 week ahead}
\centering
\resizebox{\textwidth}{!}{
\input{\PathToOutputTables/loan_utilisation_ratio_novelty_change_5.tex}
}
\label{table:loan_utilisation_ratio_novelty_change_5.tex}
\end{table}

\begin{table}[H]
\caption{Summary Stats for changes in Loan Utilisation Ratio for Novelty Level - 1 month ahead}
\centering
\resizebox{\textwidth}{!}{
\input{\PathToOutputTables/loan_utilisation_ratio_novelty_change_26.tex}
}
\label{table:loan_utilisation_ratio_novelty_change_26.tex}
\end{table}

\begin{table}[H]
\caption{Summary Stats for changes in Loan Utilisation Ratio for Reach Level - 1 week ahead}
\centering
\resizebox{\textwidth}{!}{
\input{\PathToOutputTables/loan_utilisation_ratio_reach_change_5.tex}
}
\label{table:loan_utilisation_ratio_reach_change_5.tex}
\end{table}

\begin{table}[H]
\caption{Summary Stats for changes in Loan Utilisation Ratio for Reach Level - 1 month ahead}
\centering
\resizebox{\textwidth}{!}{
\input{\PathToOutputTables/loan_utilisation_ratio_reach_change_26.tex}
}
\label{table:loan_utilisation_ratio_reach_change_26.tex}
\end{table}

\begin{table}[H]
\caption{Summary Stats for changes in Loan Utilisation Ratio for Severity Level - 1 week ahead}
\centering
\resizebox{\textwidth}{!}{
\input{\PathToOutputTables/loan_utilisation_ratio_severity_change_5.tex}
}
\label{table:loan_utilisation_ratio_severity_change_5.tex}
\end{table}

\begin{table}[H]
\caption{Summary Stats for changes in Loan Utilisation Ratio for Severity Level - 1 month ahead}
\centering
\resizebox{\textwidth}{!}{
\input{\PathToOutputTables/loan_utilisation_ratio_severity_change_26.tex}
}
\label{table:loan_utilisation_ratio_severity_change_26.tex}
\end{table}

%%%%%%%%%%%%%%%%%%%%%%%%%%%%%%%%%%%%%%%%%%%%%%%%%%%%%%%%%%%%%%%%%%%%%%%%
\subsubsection{Changes in Loan Fee}

\begin{table}[H]
\caption{Summary Stats for changes in Loan Fee for Environmental Level - 1 week ahead}
\centering
\resizebox{\textwidth}{!}{
\input{\PathToOutputTables/loan_fee_environment_change_5.tex}
}
\label{table:loan_fee_environment_change_5.tex}
\end{table}

\begin{table}[H]
\caption{Summary Stats for changes in Loan Fee for Environmental Level - 1 month ahead}
\centering
\resizebox{\textwidth}{!}{
\input{\PathToOutputTables/loan_fee_environment_change_26.tex}
}
\label{table:loan_fee_environment_change_26.tex}
\end{table}

\begin{table}[H]
\caption{Summary Stats for changes in Loan Fee for Social Level - 1 week ahead}
\centering
\resizebox{\textwidth}{!}{
\input{\PathToOutputTables/loan_fee_social_change_5.tex}
}
\label{table:loan_fee_social_change_5.tex}
\end{table}

\begin{table}[H]
\caption{Summary Stats for changes in Loan Fee for Social Level - 1 month ahead}
\centering
\resizebox{\textwidth}{!}{
\input{\PathToOutputTables/loan_fee_social_change_26.tex}
}
\label{table:loan_fee_social_change_26.tex}
\end{table}

\begin{table}[H]
\caption{Summary Stats for changes in Loan Fee for Governance Level - 1 week ahead}
\centering
\resizebox{\textwidth}{!}{
\input{\PathToOutputTables/loan_fee_governance_change_5.tex}
}
\label{table:loan_fee_governance_change_5.tex}
\end{table}

\begin{table}[H]
\caption{Summary Stats for changes in Loan Fee for Governance Level - 1 month ahead}
\centering
\resizebox{\textwidth}{!}{
\input{\PathToOutputTables/loan_fee_governance_change_26.tex}
}
\label{table:loan_fee_governance_change_26.tex}
\end{table}

\begin{table}[H]
\caption{Summary Stats for changes in Loan Fee for Novelty Level - 1 week ahead}
\centering
\resizebox{\textwidth}{!}{
\input{\PathToOutputTables/loan_fee_novelty_change_5.tex}
}
\label{table:loan_fee_novelty_change_5.tex}
\end{table}

\begin{table}[H]
\caption{Summary Stats for changes in Loan Fee for Novelty Level - 1 month ahead}
\centering
\resizebox{\textwidth}{!}{
\input{\PathToOutputTables/loan_fee_novelty_change_26.tex}
}
\label{table:loan_fee_novelty_change_26.tex}
\end{table}

\begin{table}[H]
\caption{Summary Stats for changes in Loan Fee for Reach Level - 1 week ahead}
\centering
\resizebox{\textwidth}{!}{
\input{\PathToOutputTables/loan_fee_reach_change_5.tex}
}
\label{table:loan_fee_reach_change_5.tex}
\end{table}

\begin{table}[H]
\caption{Summary Stats for changes in Loan Fee for Reach Level - 1 month ahead}
\centering
\resizebox{\textwidth}{!}{
\input{\PathToOutputTables/loan_fee_reach_change_26.tex}
}
\label{table:loan_fee_reach_change_26.tex}
\end{table}

\begin{table}[H]
\caption{Summary Stats for changes in Loan Fee for Severity Level - 1 week ahead}
\centering
\resizebox{\textwidth}{!}{
\input{\PathToOutputTables/loan_fee_severity_change_5.tex}
}
\label{table:loan_fee_severity_change_5.tex}
\end{table}

\begin{table}[H]
\caption{Summary Stats for changes in Loan Fee for Severity Level - 1 month ahead}
\centering
\resizebox{\textwidth}{!}{
\input{\PathToOutputTables/loan_fee_severity_change_26.tex}
}
\label{table:loan_fee_severity_change_26.tex}
\end{table}

%%%%%%%%%%%%%%%%%%%%%%%%%%%%%%%%%%%%%%%%%%%%%%%%%%%%%%%%%%%%%%%%%%%%%%%%

\doublespacing
\section{Conclusion}

In conclusion, we have examined the intricate relationship between ESG incidents and the landscape of securities lending, focusing on specific lending metrics for the US Stock Market. The project harnessed a rich array of data, merging intricate datasets from the likes of WRDS — including RepRisk, IHS Markit, and CRSP — to construct a robust analytical framework. Despite the hurdles of data integration, the study provided compelling insights into the conduct of short sellers in the face of ESG events.

The analysis yielded clear indications that ESG developments have a tangible impact on securities lending, evidenced by the variations in the Short Interest Ratio amidst ESG incidents. The data definitely gives some insights on the responsiveness of short sellers to ESG-related news, and opens some interesting fields for further study of these computed metrics and any other metric that would give more information on the behavior of short sellers in the US Stock Market.

Contrasting behaviors emerged when examining the lending indicators of distinct companies like Apple Inc., GameStop Corp., and Altria Group Inc. This contrast is particularly illuminating, as it highlights how company-specific factors — such as industry segment, market size, and the nature of ESG controversies — are reflected in securities lending metrics.

While the research has indeed cast light on the relationship between ESG events and market reactions, it is not without its limitations. The study's observational design and the vast spectrum of ESG incidents considered make it challenging to draw definitive conclusions about causation. Furthermore, the fast-paced evolution of both ESG reporting standards and market sentiment presents an ever-shifting landscape that this study could only snapshot.

In summation, this research underscores the growing importance of ESG considerations in the financial domain, especially concerning the actions of short sellers in the securities lending market. It lays the groundwork for further academic inquiry and suggests a future direction where further studies and more sophisticated econometric techniques might unravel the complex web of causality that underpins the observed market behaviors.

\end{document}
